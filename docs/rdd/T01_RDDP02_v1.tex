\documentclass{article}

\usepackage{graphicx}
\usepackage{amsmath}
\usepackage{caption}
\usepackage{subcaption}
\usepackage{listings}
\usepackage[hidelinks]{hyperref}
\usepackage{enumitem}
\usepackage{geometry}
\usepackage{biblatex}

\renewcommand{\contentsname}{Table of Contents}
\addbibresource{rdd.bib}
\geometry{
 a4paper,
 left=20mm,
 right=20mm,
 top=20mm,
 bottom=25mm,
 }

\begin{document}

\begin{titlepage}
\begin{center}
\vspace*{1cm}

\Huge
\textbf{Project 02: Voting Simulation}

\vspace{0.5cm}
\Large
\textit{Requirements Definition Document} \\
\textit{RDD Version 1}

\vspace{1cm}

\textbf{Team 01}

\vspace{0.5cm}

\text{Marina Seheon (Manager)} \\
\text{Andrei Phelps (Document Manager)} \\
\text{Luke McDougall (Lead Software Engineer)} \\
\text{Jack Vanlyssel} \\
\text{Spoorthi Menta} \\
\text{Vamsi Krishna Singara} \\

\vspace{1cm}

\begin{figure}[h]
    \centering
    \includegraphics[width=0.25\textwidth]{docs/rdd/figures/ballot_icon.png}
    \caption*{Image courtesy of Flaticon.com \cite{flaticonElectionsFree}.}
    \label{fig:safeIcon}
\end{figure}

\vspace{7cm}

\Large
\textbf{CS460: Software Engineering} \\

\end{center}
\end{titlepage}

\newpage

\tableofcontents

\newpage

\section{Introduction}
This document provides a thorough insight into the New Mexico Voting Simulation – a cutting-edge digital platform designed to mirror an accurate, secure, and transparent voting experience in New Mexico. Built in strict alignment with the New Mexico Election Code, this project adheres to the detailed technical and operational standards set by the state, ensuring that the voting systems for casting and counting ballots either match or exceed established state benchmarks. These standards span a wide spectrum, covering technical guidelines, operational practices, memory and storage criteria, protocols for ballot handling and processing, and strict measures for source code custody. \\ \\
Moreover, the New Mexico Voting Simulation strictly conforms to the state's Ballot Printing System Standards, ensuring proficiency in ballot preparation, robust security measures, and specific hardware, software, and user interface guidelines. These comprehensive directives guarantee that the simulation not only reflects the logistical nuances of real-world voting events but also upholds the quality and trustworthiness required by the state's regulatory framework. \\ \\
Integrating leading-edge technology and innovative features, this simulation system meets the established criteria, boasting features like unique audit trail serial numbers, protective measures for equipment, enhanced mobility for heavy apparatus, and independent operation to ensure security and reliability. The system's modular technological foundation is geared for future updates, in line with evolving guidelines set by the U.S. Election Assistance Commission and state laws. \\ \\
The document is structured into well-defined sections to offer a comprehensive insight into the New Mexico Voting Simulation initiative. We start with \textit{Regulations}, outlining the essential compliance criteria the simulation adheres to. This is followed by \textit{Objectives} which emphasizes our commitment to capturing genuine election processes, \textit{System Organization} elaborates on the principal architectural components, \textit{Capabilities} highlights the system's secure and user-centric features, \textit{Design Constraints} addresses the challenges and solutions faced in complying with the state's stringent election code, and finally, the \textit{Definition of Terms} which provides clarity on technical terminology. \\ \\
Despite facing these challenges, the project's primary objective remains unwavering: to offer a clear and robust voting simulation environment. Whether you are an election official, an active participant in the election process, or an observer, this manual provides a comprehensive perspective, positioning the New Mexico Voting Simulation as a forward-thinking, compliant, and trustworthy approach for replicating elections, adhering to the standards and procedures set forth by New Mexico's election overseers.

\section{Regulations}
The regulations section delves deep into the intricacies of the New Mexico Voting Simulation, offering readers an in-depth understanding of its operational structure. From rigorous tabulator certification requirements to detailed ballot printing system necessities \cite{nmVotingSystem}, this section serves as a cornerstone for stakeholders to understand the meticulous measures undertaken to simulate the electoral process.

\subsection{Tabulator Certification Requirements}
This segment focuses on the stringent standards governing voting tabulators. It encapsulates vital aspects such as hardware and software criteria, data handling, and contingency planning. This ensures that the tabulators are not only compliant with the prevailing legal framework but also resilient enough to handle diverse operational scenarios.

\subsubsection{Technical Requirements (1-9-7.7)}
This section stipulates the physical and functional specifications voting systems must meet. These include unique identification features, environmental protection, mobility standards, independence from network connections for security, adaptability to legal and technological changes, provisions for necessary peripherals, transparency measures, as well as specific functionalities for recording, reporting, and verifying election data.

\begin{itemize}
    \item Unique embedded internal serial number for auditing.
    \item Stand-alone, non-networked system with event recording.
    \item Scalable technology compliant with federal standards and state law.
    \item Public display of the number of ballots processed.
    \item Print capabilities for verification at the start and end of polls, compliant with state law.
    \item Feature for electronic data file report generation.
\end{itemize}

\subsubsection{Operational Requirements (1-9-7.8)}
Operational mandates focus on the software and functional reliability of voting systems. Requirements encompass specialized election software, comprehensive self-check protocols, accurate time-keeping mechanisms, and robust power backup solutions. These measures ensure the consistent, transparent, and uninterrupted operation of the voting process.

\begin{itemize}
    \item Election-specific internal application software.
    \item Comprehensive diagnostics for failure detection.
    \item Real-time clock for polling times.
    \item Internal backup battery with minimum two-hour operation during outages, without data loss.
\end{itemize}

\subsubsection{Memory and Storage Requirements (1-9-7.9)}
These provisions address the security, integrity, and reliability of data storage mechanisms in voting systems. Voting systems must utilize secure removable storage media, protect data integrity, prohibit unauthorized code execution, and facilitate thorough pre-election testing. They must also maintain detailed, retrievable audit logs and support secure methods for transmitting official results.

\begin{itemize}
    \item Programmable with removable storage media.
    \item Secure and accurate vote storage on removable devices.
    \item No executable code from RAM; nonvolatile memory for the operating system with quality checks.
    \item Pre-election testing with results storage and printout capabilities.
    \item Internal audit trails for all election events, with print support.
    \item Secure remote transmission of results post-poll closing.
    \item Data protection during report generation and transmission.
\end{itemize}

\subsubsection{Ballot Handling and Processing (1-9-7.10)}
This part outlines the specifications for how voting systems should accept and process ballots. Systems must accommodate a variety of ballot sizes, styles, and orientations, prevent over-voting, and be capable of handling the ballot complexities of large election jurisdictions. Furthermore, they must be adept at reading numerous voting positions, ensuring each vote is accurately registered.

\begin{itemize}
    \item Accommodation for ballots of specified dimensions and printing styles.
    \item Capability to accept ballots in any orientation.
    \item Overvote rejection.
    \item Accommodation for maximum ballot styles/variations in large jurisdictions.
    \item Reading of single ballots with at least 420 voting positions.
\end{itemize}

\subsubsection{Source Code Escrow (1-9-7.11)}
To safeguard the continuity and integrity of the voting process, this section requires the escrow of a voting system's source code. This measure protects against disruptions caused by the dissolution or withdrawal of support by the system's manufacturer, ensuring the state has unfettered access to the source code for legal and operational scrutiny or system maintenance.

\begin{itemize}
    \item Mandatory escrow of the voting system's source code.
    \item State access in cases where the manufacturer is defunct or no longer supports the system.
\end{itemize}

\subsection{Ballot Printing System Requirements}
A pivotal aspect of the electoral process, the ballot printing system requirements emphasize the imperative for an efficient and secure system. This encompasses provisions for early voting, requisite measures for ballot preparation, rigorous security protocols, and specifics for hardware, software, and usability. The overarching aim is to ensure a seamless, transparent, and secure voting experience for every participant.

\subsubsection{Early Voting (1-6-5.7(D))}
This section highlights the identification requirements for early voters, detailing the procedure if ID isn't presented and noting the designation for those who've voted early.

\begin{itemize}
    \item When voting at an early voting location, the voter shall provide the required voter identification to the election board, county clerk or the clerk's authorized representative.
    \item If the voter does not provide the required voter identification, the voter shall be allowed to vote on a provisional ballot. If the voter provides the required voter identification, the voter shall be allowed to vote after subscribing an application to vote on a form approved by the secretary of state or its electronic equivalent approved by the voting system certification committee.
    \item The county clerk or the clerk's authorized representative shall make an appropriate designation on the signature roster or register next to the voter's name indicating that the voter has voted early.
\end{itemize}

\subsubsection{Systems Designed to Print Ballots at Polling Locations; Ballot Preparation Requirements (1-9-20)}
This section describes the general capabilities required for systems designed to print ballots at polling locations, detailing various specifications for ballot creation.

\begin{itemize}
    \item Systems designed to print ballots at polling locations shall provide the general capabilities for ballot preparation and shall be capable of:
    \begin{itemize}
        \item[a.] Enabling the automatic formatting of ballots in accordance with the requirements of the Election Code, as amended from time to time, for offices, candidates and questions qualified to be placed on the ballot for each political subdivision and election district;
        \item[b.] Supporting the maximum number of potentially active voting positions;
        \item[c.] Generating ballots for a primary election that segregate the choices in partisan contests by party affiliation;
        \item[d.] Ensuring that voting response fields properly align with the specific candidate names or questions printed on the ballot;
        \item[e.] Generating ballots that can be tabulated by all certified voting systems in the state;
        \item[f.] Generating a ballot for an individual voter based on voter registration data provided by state or county;
        \item[g.] Functionality in absentee, early and election day voting environments;
        \item[h.] Providing absentee ballot tracking ability.
    \end{itemize}
\end{itemize}

\subsubsection{Systems Designed to Print Ballots at Polling Locations; Security Requirements (1-9-21)}
This section underscores the security prerequisites for ballot printing systems at polling locations, focusing on audit trails, activity logs, and user security levels.

\begin{itemize}
    \item Systems designed to print ballots at polling locations shall provide the security capabilities for ballot preparation and shall be capable of:
    \begin{itemize}
        \item[a.] Providing a full audit trail of individual voter activity;
        \item[b.] Providing full ballot production audit logs for all activity, including absentee voting by mail, in-person absentee voting, early voting, provisional voting and spoiling ballots;
        \item[c.] Creation and preservation of an audit trail of every ballot issued, including during a period of interrupted communication in the event of loss of network connectivity;
        \item[d.] Suitable security passwords at user, administrator and management levels.
    \end{itemize}
\end{itemize}

\subsubsection{Systems Designed to Print Ballots at Polling Locations; Hardware, Software and Usability Requirements (1-9-22)}
This section lays out the hardware, software, and usability requisites for ballot printing systems, emphasizing integration with voter registration systems and user-friendly interfaces.

\begin{itemize}
    \item Systems designed to print ballots at polling locations shall:
    \begin{itemize}
        \item[a.] Provide hardware requirements that:
        \begin{itemize}
            \item[i.] Shall be networkable and scalable for multi-user environments;
            \item[ii.] Contain prominent instructions as to any special requirements.
        \end{itemize}
        \item[b.] Provide software requirements that:
        \begin{itemize}
            \item[i.] Be capable of exporting voter data and voter activity status data to state and county voter registration systems;
            \item[ii.] Be capable of generating all required absentee and early voting signature rosters in a state-approved format;
            \item[iii.] Generate daily and to-date activity reports based on user-defined criteria;
            \item[iv.] Have both single transaction and batch transaction absentee production capability;
        \end{itemize}
        \item[c.] Be capable of being operated by computer users familiar with a graphical user interface.
    \end{itemize}
\end{itemize}

\section{Objectives}
In the ever-evolving landscape of electoral technology, the New Mexico Voting Simulation aspires to bridge the gap between traditional voting methods and modern digital solutions. With an emphasis on accessibility, accuracy, and trust, our objectives are meticulously designed to enhance the democratic process, while ensuring the sanctity of each vote remains uncompromising.

\subsection{Facilitate Transparent Elections}
By deploying advanced digital platforms backed by blockchain and encryption technologies, our objective is to ensure that every vote cast is transparent, verifiable, and remains uncompromised. This initiative seeks to eradicate voting discrepancies and provides a clear, immutable record of every ballot cast.

\subsection{Achieve State Compliance}
Adherence to the New Mexico Election Code and other state guidelines is paramount. Beyond mere compliance, we aim to develop a system that serves as a gold standard, setting benchmarks for electoral systems both nationally and internationally.

\subsection{Promote Public Confidence}
In an age of skepticism and misinformation, ensuring the public's trust in electoral systems is vital. Through rigorous testing, in-depth system audits, and open channels of communication, we strive to foster public trust, assuring voters that their voice is genuinely, transparently, and accurately represented.

\subsection{Ensure Security and Reliability}
In a world rife with cyber threats, security is a top priority. Our system is fortified with multi-layered cybersecurity protocols. Our goal is to implement a foolproof system, resistant to external threats, manipulations, and equipped with real-time monitoring and immediate disaster recovery capabilities.

\subsection{Encourage Voter Participation}
The foundation of any democracy is its electorate. By providing an intuitive, user-friendly interface coupled with extensive voter education campaigns, and ensuring that the voting process is quick and straightforward, we aim to motivate and empower more citizens to participate actively in the democratic process.

\section{System Organization}
Blah blah blah...

\section{Capabilities}
Blah blah blah...

\section{Design Constraints}
Blah blah blah...

\section{Definition of Terms}
This section provides definitions for critical terms recurrently utilized throughout the document. This section can be a reference point for readers engaging with the content.

\begin{enumerate}
    \item[I.] \textbf{Item 1}: Wow, so cool.
    \item[II.] \textbf{Item 2}: Wow, so cool.
    \item[III.] \textbf{Item 3}: Wow, so cool.
    \item[IV.] \textbf{Item 4}: Wow, so cool.
    \item[V.] \textbf{Item 5}: Wow, so cool.
\end{enumerate}

\newpage

\printbibliography

{\parindent0pt}

\end{document}
